\section{Einleitung}
    Die automatische Klassifikation von Dokumenten spielt in der modernen Welt eine tragende Rolle. 
    Wegen des rasanten Wachstums an Informationen im Internet ist es notwendig, Dokumente bezüglich ihrer Art oder ihres Inhalts in Kategorien zu unterteilen, um effiziente Suchen zu ermöglichen. 

    Einen großen Anwendungsbereich stellt auch der tägliche Umgang mit Diensten wie zum Beispiel Email dar, bei der eine Klassifikation von Text unabdingbar ist, um wichtige Nachrichten von unnötiger Werbung zu trennen (Spam-Filterung).
    Ebenso ist es möglich, den eigenen Posteingang automatisch nach Kategorien sortieren zu lassen~\cite{IIR}.

    Eine weitere mögliche Nutzung von Klassifikatoren ist die Filterung von Nachrichten, um dem Nutzer nur für ihn ``interessante'' Nachrichten zu präsentieren. 
    In~\cite{PersoNews} wurde ein solches System implementiert.\\

    Ziel dieser Arbeit ist es, die automatische Klassifikation von Textdokumenten nach Themen mit Hilfe eines naiven Bayes-Klassifikators zu erarbeiten.

    Kapitel 2 behandelt dazu die Grundlagen der Bayesschen Statistik, welche auch im naiven Bayes-Klassifikator zur Anwendung kommen. 

    Die theoretischen Hintergründe und Prinzipien des Klassifikators werden in Kapitel 3 behandelt, bevor in Kapitel 4 dann eine Implementierung der theoretischen Grundlagen genauer betrachtet wird sowie auf die weiteren Verarbeitungsschritte der Textklassifikation eingegangen wird.
