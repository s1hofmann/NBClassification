\section{Fazit}
    Anhand dieser Arbeit wurden die theoretischen Grundlagen des naiven Bayes-Klassifikators erarbeitet, sowie die Implementierung eines Klassifikators behandelt. 
    Dadurch ließ sich erkennen, dass sich ein Bayes-Klassifikator leicht implementieren lässt und mit 88\% Genauigkeit gute Ergebnisse liefert.

    Durch die Verwendung eines Vektorraummodells mit z.B. tf-idf Werten~\cite{IIR}, anstatt des in der vorliegenden Implementierung verwendeten Unigram-Wortsequenz-Modells, ließe sich die Genauigkeit des Klassifikators womöglich noch verbessern.

    %Da das verwendete Unigram-Wortsequenz-Modell stark von der Länge der einzelnen Dokumente abhängt, liese sich die Genauigkeit der Klassifikation eventuell durch die Verwendung eines Vektorraum-Modells verbessern.
