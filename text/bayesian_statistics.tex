\section{Bayessche Statistik}\label{sec:statistics}
    In diesem Kapitel sollen Grundbegriffe der Bayesschen Statistik eingeführt werden, welche in dieser Arbeit Verwendung finden. 
    Hierbei handelt es sich keineswegs um eine vollständige Einführung in die Bayessche Statistik, vielmehr werden die für diese Arbeit wichtigsten Aspekte zusammengefasst.
    Der Inhalt des Kapitels stützt sich dabei auf~\cite{bw_statistik}.
    
    \subsection{Wahrscheinlichkeitsverteilung}
        Existiert eine Abbildung $P$, welche einem Ereignis $A$ aus einer Ereignismenge $\Omega$ eine Zahl $P(A)$ zuweist, so spricht man von einer Wahrscheinlichkeitsverteilung, sofern die so genannten Kolmogorov-Axiome gelten:
        \begin{enumerate}
            \item $0 \le P(A) \le 1$
            \item $P(\Omega) = 1$
            \item $P(A_{1} \cup A_{2} \cup \dots \cup A_{n}) = \sum_{i=1}^{n}P(A_{i})$, wenn $A_{1}, \dots, A_{n}$ disjunkt.
        \end{enumerate}

    \subsection{Bedingte Wahrscheinlichkeit}
        Als bedingte Wahrscheinlichkeit $P(X|Y)$~wird die Wahrscheinlichkeit bezeichnet, dass Bedingung $X$ erfüllt ist, gegeben dem Fall, dass eine Bedingung $Y$ ebenfalls erfüllt ist.

        \begin{equation}
            P(X|Y) = \frac{P(X \cap Y)}{P(Y)}
            \label{eq:cond}
        \end{equation}

        \begin{equation}
            \Leftrightarrow P(X \cap Y) = P(X|Y)P(Y)
            \label{eq:cap}
        \end{equation}

    \subsection{Stochastische Unabhängigkeit}
        Gelten zwei Ereignisse $X$ und $Y$ als stochastisch unabhängig, so haben sie keinen Einfluss aufeinander.
        Das Eintreten von Ereignis $Y$ hat somit keinerlei Auswirkung auf die Wahrscheinlichkeit des Eintretens von Ereignis $X$.

        Es gilt damit:

        \begin{equation}
            P(X|Y) = P(X) \Leftrightarrow P(Y|X) = P(Y)
            \label{eq:independence}
        \end{equation}

    \subsection{Satz von Bayes}
        Um aus einer gegebenen bedingten Wahrscheinlichkeitsverteilung $P(X|Y)$ und einer ebenfalls bekannten Wahrscheinlichkeit $P(Y)$ die bedingte Wahrscheinlichkeit $P(Y|X)$ zu berechnen, findet der Satz von Bayes Anwendung. 
        Ausgehend von Gleichungen~\ref{eq:cond} und~\ref{eq:cap} lässt sich folgender Zusammenhang feststellen:

        \begin{equation}
            P(Y|X) = \frac{P(Y \cap X)}{P(X)}
            \label{eq:bayes_1}
        \end{equation}

        Mit $P(Y \cap X) = P(X|Y)P(Y)$ ergibt sich daraus Gleichung~\ref{eq:bayes_2}, welche als Satz von Bayes bezeichnet wird:

        \begin{equation}
            P(Y|X) = \frac{P(X|Y)P(Y)}{P(X)}
            \label{eq:bayes_2}
        \end{equation}
